documentclass{article}

\begin{document}
% start of titlepage -- {{{
\begin{titlepage}
	\centering
	{\scshape\Huge Waterloo Rocketry Team \par}
	\vspace{1.5cm}
	{\scshape\bfseries\LARGE Electrical Design and Assembly Standards\par}
	\vspace{2cm}
	{\Large\itshape Aaron Morrison\par}
	\vfill

% Bottom of the page
	{\large July 3, 2017\par Revision 0.2}
    \par
\end{titlepage}
% end titlepage -- }}}

%start of guiding principles section -- {{{
\section{Guiding Principles of all decisions}
This document exists to set out a standard for all the electrical gear (avionics, telemetry, ground support equipment). This document is very much still in development (it's based on 3 pages of notes I made in a notebook in the desert), and I am by no means an expert. Changes/revisions are very welcome. All design decisions on mission critical electrical systems should be made with the following four points in mind:

\begin{enumerate}
\item Our electrical stuff needs to work 100\% of the time at competition. If that's unattainable, 6 sigma is also acceptable. A lot of work from a lot of people go into building this rocket, and missing our 1 launch attempt because of an electrical problem is not acceptable.
\item Once we leave the bay, all electrical work should be done. Fixing something in the desert requires running a generator in a trailer in 40 degree heat. No one does their best work in 40 degree heat. Any MacGyvered solution you come up with there will be orders of magnitude worse than a proper solution you create in Waterloo, and if you're in the desert, there are better things you should be spending your time on.
\item The rest of the team should never be waiting on a fix from us. Electrical stuff should work, and it should work as soon as we need it to. With electrical systems we have the luxury to make sure they work in advance (the engine team does not have this), and we should know all the failure modes and be prepared for them. 
\item We should never be asking another university's team for gear in the desert. Whether or not we fly should not be at the whim of whether we can find something we could have packed a spare of. If something can break, you should have a spare. If there's a tool you need, you should have it with you. We bring everything we need; we shouldn't be asking around for gear.
\end{enumerate}
This document is written in descending order of importance. If you need to decide between whether to follow a guiding principle or a hard rule, follow the principle. As long as your decisions demonstrably don't violate any of these 4, you're probably ok.
%end of guiding principles section -- }}}

%start of hard rules section -- {{{
\section{Rules}
Here are the rules for all the design and assembly decisions. If you want to break one of these rules, make sure you have a very good excuse. Again though, this document was written by someone who doesn't actually have any idea of what he's doing, so if you think a rule is stupid, feel free to cross it out (provide a reason though). If you think a rule is unclear, feel free to elaborate on it. If you think there's a good rule that should be added, please write it in in the margin or pull request the repository (https://github.com/akmorrison/waterloo_rocketry_electrical_standard).

\begin{itemize}

\item Don't solder stranded wire to protoboard. Don't solder stranded wire to a connector. It gets brittle and it breaks.
\item Don't crimp something onto solid core wire. That doesn't work
\item All mission critical components should have a hotswappable replacement. You should have that replacement's location written down.
\item Hotswappable means you can replace it without any tools (except maybe pliers or a screwdriver. But no generator)
\item All exposed wire should be heatshrunk, all heatshrink should be clear.
\item If supplies allow, use different a different insulation color for each wire (at the very least, positive should be red, ground should be black, and all other wires should not be red or black)
\item If you screw up a soldering job (we all do it), redo the job
\item All connections must be tug proof.
\item If your system has any software anywhere, that software should be unit tested
\item All wires you use need to be low enough gauge to match the current you're putting through them
\item If you're using an RC lithium polymer battery, it should be fused (those fuckers can sink 50 amps)
\item If it's conceivably possible for a component to be installed backwards, there should be idiot proof documentation of which way it goes, that documentation should be less than 2 inches from the install point
\item If a connection could be asymmetric, it should be asymmetric
\item All switches need idiot proof labelling
\item But don't use actual tape labels (they peel off easy. Use a sharpie)
\item All mission critical software should be expressible (and accompanied by) less than 100 lines of pseudocode.
\item If your wires get handled or moved or torqued at any point, it should be strain relieved on both sides of the torque point
\item If you absolutely need to solder stranded core to something, make sure it's very strain relieved (I personally like P clamps, but zip ties work too)
\item Tight wires between two points are accidents waiting to happen
\item Strain relieve everything to the point that it can survive violent shaking (if it's actually going in the rocket, it should survive very violent shaking)
\end{itemize}
%end of hard rules section -- }}}

%start of documentation section -- {{{
\section{Documentation Standards}
If you're building some piece of gear, or you're leading a project that does, it is your responsibility to ensure that all the forms of documentation are written decently and put in the google drive.

There are three types of documentation that should be written for every piece of electrical gear that comes to competition:

\begin{enumerate}
\item Users guide
\begin{itemize}
\item Step by step operating instructions
\item Partway through a procedure, if something can go wrong, say what it is, how you can tell it's gone wrong, and a reference to your troubleshooting guide on how to fix it.
\item This probably just needs to be a single checklist/procedure
\end{itemize}
\item Troubleshooting Guide
\begin{itemize}
\item detail every failure mode you can think of (and try to think of more)
\item include the symptoms of every failure mode, what the worst case scenario is, how to diagnose this failure mode conclusively, and how to fix it.
\item this should have a list of every component in your system, and how to test that component in isolation. This list will be useful for building spares kits.
\end{itemize}
\item Development Logs
\begin{itemize}
\item Every meeting's minutes, every block diagram you draw, all your rambling dev notes, everything you do pertaining to this system during the development cycle.
\item Take all this, pick good filenames, and stick them all in a folder in the google drive.
\item This will be incredibly useful for future generations who want to learn this stuff.
\item Include data sheets for everything you use.
\end{itemize}
\end{enumerate}
%end of documentation section -- }}}

%start of life advice section -- {{{
\section{Life advice from someone who doesn't actually know what they're doing}
\begin{itemize}
\item Buy spares. You will fuck things up, be prepared
\item Never solder at hotter than 700 degrees. Even that is probably pretty aggressive. Use leaded solder, it melts at a low temperature so you won't burn off all your flux.
\item All software should use a VCS (probably git), and all final versions of your software should go with your dev logs on the drive.
\item Don't short things. If you have any bare wires flailing around, don't apply power to the system.
\item Assume that you will not be there when your system is used. Your documentation should be so good that someone who has never seen your system can fix/use it
\item Wires and electrical components do take up space. Please keep this in mind.
\item Keep good isolation. That means standoffs and that means decent creepage distances.
\end{itemize}
%end of life advice section -- }}}
\end{document}
